\subsection{Завдання 1:}

\textbf{Біфуркація Хопфа}
\label{pr5:1}

$\left\{\begin{aligned}
    \dot{x}_1 = -\Gamma x_1 - ax_2 - x_1r^2\\
    \dot{x}_2 = -\Gamma x_2 + ax_1 - x_2r^2
\end{aligned}\right.$

\textbf{В полярні координати:} 


$\left\{\begin{aligned}
    x_1 = r\cos\varphi \\
    x_2 = r\sin\varphi
\end{aligned}\right.$

$\left\{\begin{aligned}
    \dot{r}\cos\varphi - r\dot{\varphi}\sin\varphi &= -\Gamma r\cos\varphi - ar\sin\varphi - r^3\cos\varphi \ \ (*)\\
    \dot{r}\sin\varphi + r\dot{\varphi}\cos\varphi &= -\Gamma r\sin\varphi + ar\cos\varphi - r^3\sin\varphi \ \ (**)
\end{aligned}\right.$

$\dot{r}:\ \ (*)\cos\varphi + (**)\sin\varphi \Longrightarrow \dot{r} = -\Gamma r - r^3$

$\dot{\varphi}:\ \ \frac{(*)(-\sin\varphi) + (**)\cos\varphi}{r} \Longrightarrow \dot{\varphi} = a$


\begin{equation} \label{pr5:tsk1_1}
    \left\{\begin{aligned}
        \dot{r} &= -r(\Gamma + r^2)\\
        \dot{\varphi} &= a
    \end{aligned}\right.
\end{equation}

\textbf{Розв'язати систему (\ref{pr5:tsk1_1})}

$-\frac{dr}{r(\Gamma + r^2)} = dt$

заміна: $r^2 = z, \ dr = \frac{1}{2}z^{-\frac{1}{2}} dz $

$-\frac{1}{2}\frac{dx}{z(\Gamma + z)} = dt$

$\Bigg | \frac{A}{z} + \frac{B}{\Gamma + z} \Longrightarrow 
\left\{\begin{aligned}
    &A\Gamma = 1\\
    &A+B = 0 
\end{aligned}\right. \Longrightarrow 
\left\{\begin{aligned}
    &A = \frac{1}{\Gamma} \\
    &B = -\frac{1}{\Gamma} \\
\end{aligned}\right|$

$-\frac{dz}{\Gamma(\Gamma + z)} + \frac{dz}{\Gamma z} = -2dt$

$-2t = \frac{1}{\Gamma}(-\ln(z+\Gamma) + \ln(z_0+\Gamma) + \ln z - \ln z_0)$

$-2\Gamma t = \ln(\frac{(z_0 + \Gamma)z}{(z+\Gamma)z_0})$

$e^{-2\Gamma t} = \frac{z_0 + \Gamma}{z_0} \frac{z}{z+\Gamma}$

$\frac{z_0}{z_0+\Gamma} e^{-2\Gamma t} = \frac{z}{z+\Gamma}$

$z = \frac{\Gamma(\frac{z_0}{z_0+\Gamma})e^{-2\Gamma t}}{1-\frac{z_0}{z_0+\Gamma}e^{-2\Gamma t}} = 
\frac{\Gamma}{\frac{z_0+\Gamma}{z_0}e^{2\Gamma t} - 1}$

$z = \frac{\Gamma z_0}{(z_0+\Gamma)e^{2\Gamma t} - z_0}$

$r = \sqrt{\frac{\Gamma r^2_0}{(r^2_0+\Gamma)e^{2\Gamma t} - r^2_0}}$

при $ \Gamma < 0$ і $t \rightarrow \infty \Longrightarrow r(t)\rightarrow \sqrt{|\Gamma|}
\Longrightarrow$ тобто існує граничнй цикл: коло із радіусом $\sqrt{|\Gamma|}$ 

цикл є стійким, оскільки при $r_0 > \sqrt{|\Gamma|}$, і при $r_0 < \sqrt{|\Gamma|}$
$r(t) \underset{t \to \infty}{\longrightarrow} \sqrt{|\Gamma|}$

\textbf{Інший спосіб виявити граничний цикл(не розв'язуючи систему Д.Р.)}

$\left\{\begin{aligned}
    &\dot{r} = -r(\Gamma + r^2)\\
    &\dot{\varphi} = a
\end{aligned}\right.$

При $\Gamma < 0$ і \\[-15mm]
\begin{itemize}
    \item {
        $r \to 0,$ можна відкинути член $-r^3$, оскільки $r\Gamma$ домінує, тоді 
        $\dot{r} = -r\Gamma > 0$ отже $r$ зростає
    }
    \item {
        при $r \to \infty$ відкидаємо $-r\Gamma$ оскільки $-r^3$ домінує, в такому разі
        $\dot{r}=-r^3 < 0$, $r$ спадає
    }
\end{itemize} 

Отже можна зробити висновок, що існує певне значення $R$ до якого прямє $r$.

\textbf{Дослідити цикл на стійкість}

Нехай $r \to R + \delta r$,  підставимо даний вираз у вихідне Д.Р, при $\Gamma < 0$.

$\frac{d}{dt}(R+\delta r) = -(R+\delta r)\Gamma - (R+\delta r)^3$

$\frac{d}{dt}\delta r = -R\Gamma -\Gamma\delta r - R^3 - 3R^2 \delta r - 
3R\delta r^2 - \delta r^3$

відкинемо члени в яких порядки більше 1, $R = \sqrt{|\Gamma|}$

$\frac{d}{dt}\delta r =$ \textcolor{red}{$-R\Gamma$} \textcolor{blue}{$-\Gamma\delta r$} 
\textcolor{red}{$-R^3$} \textcolor{blue}{$-3R^2\delta r$} 

$\frac{d}{dt}\delta r = -2|\Gamma|\delta r$

знак $\frac{d}{dt}\delta r$ протилежний до знака $\delta r$, отже цикл є стійким